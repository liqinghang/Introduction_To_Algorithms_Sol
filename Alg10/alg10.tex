\documentclass[UTF8]{ctexart}
\usepackage{algorithm}
\usepackage{algorithmic}
\usepackage{amsmath,amssymb}
\usepackage{booktabs}
\usepackage{geometry}
\usepackage{tikz}
\usepackage{color}

\geometry{a4paper,scale=0.7}

\renewcommand{\algorithmicrequire}{ \textbf{Input:}} %Use Input in the format of Algorithm
\renewcommand{\algorithmicensure}{ \textbf{Output:}} %UseOutput in the format of Algorithm

\begin{document}
    李青航 SA22225226

    \noindent\textbf{17.4-2}

    当删除后,如果$\alpha _i\ge 1/2$
    \begin{equation*}
        \begin{aligned}
            \hat{c_i}&=c_i+\Phi _i -\Phi _{i-1}\\
            &=1+(2num_i-size_i)-(2num_{i-1}-size_{i-1})\\
            &=1+2num_i-size_i-(2num_i+2-size_i)\\
            &=-1
        \end{aligned}
    \end{equation*}

    当删除后,如果$\alpha _i <1/2$ 
    \begin{equation*}
        \begin{aligned}
            \hat{c_i}&=c_i+\Phi _i -\Phi _{i-1}\\
            &=1+(\frac{1}{2}size_i-num_i)-(2num_{i-1}-size_{i-1})\\
            &=1+(\frac{1}{2}size_i-num_i)-(2num_i+2-size_i)\\
            &=-1+\frac{3}{2}size_i-3num_i
        \end{aligned}
    \end{equation*}

    因为$\alpha _i=num_i/size_i<1/2$

    \begin{equation*}
        \begin{aligned}
            \hat{c_i}&=-1+\frac{3}{2}size_i-3num_i\\
            &\le 1
        \end{aligned}
    \end{equation*}

    所以摊还代价 的上界是一个常数


    ~\\
    \noindent\textbf{17.4-3}

    首先,显然势函数$\Phi(T)>0$

    当删除后,如果$\alpha _i \ge 1/3$
    \begin{equation*}
        \begin{aligned}
            \hat{c_i}&=c_i+\Phi _i -\Phi _{i-1}\\
            &=1+|2num_i-size_i|-|2num_{i-1}-size_{i-1}|\\
            &=1+|2num_i-size_i|-|2num_i+2-size_i|\\
            &\le 1+|2num_i-size_i|-|2num_i-size_i|+2\\
            &=3
        \end{aligned}
    \end{equation*}

    当删除后,如果$\alpha _i < 1/3$ ,收缩

    此时有$num_i=num_{i-1}-1,~~size_{i}=\frac{2}{3}size_{i-1},~~2num_i=size_i$

    \begin{equation*}
        \begin{aligned}
            \hat{c_i}&=c_i+\Phi _i -\Phi _{i-1}\\
            &=num_i+1+|2num_i-size_i|-|2num_{i-1}-size_{i-1}|\\
            &=num_i+1+|2num_i-size_i|-|2num_i +2 -\frac{3}{2}size_i|\\
            &\le num_i+1+|2num_i-size_i|-|2num_i -\frac{3}{2}size_i|+2\\
            &=num_i+1-|size_i-\frac{3}{2}size_i|+2\\
            &=3+num_i-\frac{1}{2}size_i\\
            &=3
        \end{aligned}
    \end{equation*}

    所以摊还代价的上界是常数
\end{document}