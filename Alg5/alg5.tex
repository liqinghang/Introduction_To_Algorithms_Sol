\documentclass[UTF8]{ctexart}
\usepackage{algorithm}
\usepackage{algorithmic}
\usepackage{amsmath,amssymb}
\renewcommand{\algorithmicrequire}{ \textbf{Input:}} %Use Input in the format of Algorithm
\renewcommand{\algorithmicensure}{ \textbf{Output:}} %UseOutput in the format of Algorithm
% 参考:https://blog.csdn.net/jzwong/article/details/52399112
\begin{document}
李青航 SA22225226\\

\noindent\textbf{14.3-2}

\begin{algorithm}
	\caption {INTERVAL-SEARCH$(T,i)$}
	\label{alg:1}
	\begin{algorithmic}[1]
		\STATE $x=T.root$
		\WHILE{$x \ne T.nil$ and $i$ does not overlap $x.int$}
			\IF{$x.left \ne T.nil$ and $x.left.max > i.low$}
				\STATE $x=x.left$
			\ELSE
				\STATE $x=x.right$
			\ENDIF
		\ENDWHILE
		\RETURN $x$
		
	\end{algorithmic}
\end{algorithm}

\noindent\textbf{15.1-4}

见算法2,3. 
数组$s$保存,$s[i]$表示长度为$i$的钢条,从哪里切
\begin{algorithm}
	\caption {MEMOIZED-CUT-ROD$(p,n)$}
	\label{alg:2}
	\begin{algorithmic}[1]
		\STATE let $r[0...n]$ be a new array
		\STATE let $s[0...n]$ be a new array
		\FOR{$i=0$ to $n$}
			\STATE $r[i]=-\infty$
			\STATE $s[i]=0$
		\ENDFOR
		\RETURN MEMOIZED-CUT-ROD-AUX$(p,n,r,s)$
	\end{algorithmic}
\end{algorithm}
\begin{algorithm}
	\caption {MEMOIZED-CUT-ROD-AUX$(p,n,r,s)$}
	\label{alg:3}
	\begin{algorithmic}[1]
		\IF{$r[n] \ge 0$}
			\RETURN $r[n]$
		\ENDIF
		\IF{$n==0$}
			\STATE $q=0$
		\ELSIF{$q=-\infty$}
			\FOR{$i=1$ to $n$}
				\STATE $t=p[i]+$MEMOIZED-CUT-ROD-AUX$(p,n-i,r,s).q$
				\IF{$t\ge q$}
					\STATE $q=t$
					\STATE $s[n]=i$
				\ENDIF
			\ENDFOR
		\ENDIF
		\STATE $r[n]=q$
		\RETURN $q$ and $s$
	\end{algorithmic}
\end{algorithm}

~\\
\noindent\textbf{15.2-1}

方案 $(A_1A_2)((A_3A_4)(A_5A_6))$

最小代价$ 5 \cdot 50 \cdot 6 + 3 \cdot 12 \cdot 5 + 5 \cdot 10 \cdot 3 + 3 \cdot 5 \cdot 6 + 5 \cdot 3 \cdot 6 =
1500 + 180 + 150 + 90 + 90 = 2010$

~\\
\noindent\textbf{15.2-2}\\
见算法4
\begin{algorithm}
	\caption {MATRIX-CHAIN-MULTIPLY$(A,s,i,j)$}
	\label{alg:4}
	\begin{algorithmic}[1]
		\IF{$i==j$}
			\RETURN $A_i$
		\ENDIF
		\RETURN MATRIX-CHAIN-MULTIPLY$(A,s,i,s[i,j])$$\cdot$ MATRIX-CHAIN-MULTIPLY$(A,s,s[i,j]+1,j)$
	\end{algorithmic}
\end{algorithm}

\noindent\textbf{15.3-1}

穷举快

穷举法根据组合数学知识,时间复杂度是一个指数复杂度。而RECURSIVE-MATRIX-CHAIN也是一个指数复杂度的,但是还有递归函数调用开销,还会重叠算某些子问题(穷举没有重叠),重复调用函数。

~\\
\noindent\textbf{15.3-2}

图太难画,略。归并排序并没有重叠子问题,每次归并,没有重复可用的操作,所以备忘技术没有用


\end{document}