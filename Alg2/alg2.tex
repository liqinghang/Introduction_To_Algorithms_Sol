\documentclass[UTF8]{ctexart}
\usepackage{algorithm}
\usepackage{algorithmic}
\usepackage{amsmath,amssymb}
\renewcommand{\algorithmicrequire}{ \textbf{Input:}} %Use Input in the format of Algorithm
\renewcommand{\algorithmicensure}{ \textbf{Output:}} %UseOutput in the format of Algorithm
% 参考:https://blog.csdn.net/jzwong/article/details/52399112
\begin{document}
李青航 SA22225226\\
\noindent\textbf{2.3-2}\\
算法见Algorithm 1,其中$A$为数组,$p$为最左游标,$r$为最右游标,$p,q,r$将数组分为区间左半边$[p,q]$,右半边$(q,r]$ .\\

\begin{algorithm}
    \caption{MERGE$(A,p,q,r)$}
    \label{alg:1}
    \begin{algorithmic}[1]
        \STATE $n_1=q-p+1$
        \STATE $n_2=r-q$
		\STATE let $L[1...n_1]$ and $R[1...n_2]$ be two new array
		\FOR{ $i=1$ to $n_1$}
			\STATE $L[i]=A[p+i-1]$
		\ENDFOR
        \FOR{$j=1$ to $n_2$}
			\STATE $L[j]=A[q+j]$
		\ENDFOR
		\STATE $i=1$
		\STATE $j=1$
		\STATE $k=p$
		\WHILE{$i \ne n_1 +1 $ and $j \ne n_2+1$}
			\IF{$L[i] \le R[i]$}
				\STATE $A[k]=L[i]$
				\STATE $i=i+1$
			\ELSIF{$A[k]=R[j]$}
				\STATE $j=j+1$
			\ENDIF
			\STATE $k=k+1$
		\ENDWHILE
		\IF {$i==n_1+1$}
			\FOR{$m=j$ to $n_2$}
				\STATE $A[k]=R[m]$
				\STATE $k=k+1$
			\ENDFOR
		\ENDIF
		\IF {$j==n_2+1$}
			\FOR{$m=i$ to $n_1$}
				\STATE $A[k]=L[m]$
				\STATE $k=k+1$
			\ENDFOR
		\ENDIF 
    \end{algorithmic}
\end{algorithm}
\noindent\textbf{2.3-3}\\
由指数运算法则$n=\lg 2^n$和条件

\begin{equation}
%\nonumber
T(n)= 
    \begin{cases}
        2, & \text{if } n==2\\
        2T(n/2)+n, & \text{if } n=2^k, k>1
    \end{cases}
\end{equation}
\\当$k=1$时,由$(1)$得$T(2)=2$
\\假设$n=2^k$时,有
\begin{equation}
T(2^k)=2T(2^k/2)+2^k
\end{equation}
\\则当$n=2^{k+1}$时,根据$(2)$,且把猜测$T(n)=n\lg n$带入有

\begin{equation}
\nonumber
\begin{split}
T(2^{k+1})&=2T(2^{k+1}/2)+2^{k+1}=2T(2^k)+2^{k+1}\\
&=2(2^k\lg(2^k))+2^{k+1}=k2^{k+1}+2^{k+1}\\&=(k+1)2^{k+1}=2^{k+1}\lg(2^{k+1})=n\lg(n)
\end{split}
\end{equation}
\\
\textbf{2.3-4}\\
\begin{equation}
\nonumber
T(n)= 
    \begin{cases}
        \Theta(1), & \text{if } n\le 1 \\
        T(n-1)+I(n), & \text{otherwise }
    \end{cases}
\end{equation}
其中,$T(n)$表示对$n$个元素的数组$A$进行插入排序的时间,$I(n)$表示将第$n$个元素插入到已经排好的$A[1...n-1]$的时间。当最坏的情况下,$I(n) \in \Theta(n)$ .
\\
\textbf{4.3-1}\\
根据所证明结论,猜测$T(n)\in O(n^2)$,所以,将$T(n)<n^2$带入:
$$
T(n)\le c(n-1)^2+n=cn^2+(1-2c)n+1 \le cn^2-2c+2 \le cn^2
$$
根据$O(n^2)$定义,证明成立
\\
\textbf{4.3-2}\\
根据所证结论知道,$T(n)\in O(\lg n)$,存在有$T(n)\le 3\lg (n)-1$,将其带入
\begin{equation}
\nonumber
\begin{split}
T(n)&=T(\lceil n/2 \rceil)+1\\
&\le 3\lg (\lceil n/2 \rceil)-1+1\\
&\le 3\lg(3n/4)\\
&= 3\lg(n)-3\lg(3/4) \\
&\le 3\lg(n)-1
\end{split}
\end{equation}
\\
\noindent\textbf{4.3-3}\\
假设有$T(n) \ge cn\lg n$\\
\begin{equation}
\nonumber
\begin{split}
T(n)&\ge 2(c\lfloor n/2 \rfloor \lg(\lfloor n/2 \rfloor))+n \\
&\ge cn\lg (n/2) +n\\
&=cn\lg n -cn\lg 2 +n\\
&=cn\lg n - cn+n\\
&\ge cn \lg n
\end{split}
\end{equation}
所以有下界,即$T(n)\in \Omega(n\lg n)$\\
又由题目知已经有上界$T(n) \in O(n\lg n)$,所以$T(n)\in \Theta(n\lg n)$
\\
\noindent\textbf{4.4-1}\\
根据递归树,得出$T(n)\in O(n^{\lg 3})$,将$T(n) \le 2n^{\lg 3}-2n$带入验证:
\begin{equation}
\nonumber
\begin{split}
T(n)&=3T(\lfloor n/2 \rfloor)+n\\
&\le 6(\frac{n}{2})^{\lg 3}-n+n\\
&= c n^{\lg 3}
\end{split}
\end{equation}
所以$T(n) \in O(n^{\lg 3})$
\\
\noindent\textbf{4.4-6}\\
根据书图4-6,每层的时间代价都是$cn$,高度为$\log_3 n$,所以总代价为:\\
$cn\log_3 n=\frac{c}{\log 3}n\lg n \in \Omega(n\lg n)$\\
\noindent\textbf{4.5-1}\\
\textbf{a.} $\Theta(n^{\frac{1}{2}})$ (情况1)\\
\textbf{b.} $\Theta(\sqrt{n}\lg n)$ (情况2)\\
\textbf{c.} $\Theta(n)$ (情况3)\\
\textbf{d.} $\Theta(n^2)$ (情况3)\\
\noindent\textbf{4.5-3}\\
套用主方法,其中$a=1, b=2$,所以$\Theta (n^{\log_2 1})=\Theta(1)$,使用情况2,答案为$\Theta(\lg n)$
\end{document}
