\documentclass[UTF8]{ctexart}
\usepackage{algorithm}
\usepackage{algorithmic}
\usepackage{amsmath,amssymb}
\renewcommand{\algorithmicrequire}{ \textbf{Input:}} %Use Input in the format of Algorithm
\renewcommand{\algorithmicensure}{ \textbf{Output:}} %UseOutput in the format of Algorithm
% 参考:https://blog.csdn.net/jzwong/article/details/52399112
\begin{document}
李青航 SA22225226\\

\noindent\textbf{15.2-1}

方案 $(A_1A_2)((A_3A_4)(A_5A_6))$

最小代价$ 5 \cdot 50 \cdot 6 + 3 \cdot 12 \cdot 5 + 5 \cdot 10 \cdot 3 + 3 \cdot 5 \cdot 6 + 5 \cdot 3 \cdot 6 =
1500 + 180 + 150 + 90 + 90 = 2010$

~\\
\noindent\textbf{15.2-2}\\
见算法1
\begin{algorithm}
	\caption {MATRIX-CHAIN-MULTIPLY$(A,s,i,j)$}
	\label{alg:1}
	\begin{algorithmic}[1]
		\IF{$i==j$}
			\RETURN $A_i$
		\ENDIF
		\RETURN MATRIX-CHAIN-MULTIPLY$(A,s,i,s[i,j])$$\cdot$ MATRIX-CHAIN-MULTIPLY$(A,s,s[i,j]+1,j)$
	\end{algorithmic}
\end{algorithm}

\noindent\textbf{15.3-1}

穷举快

穷举法根据组合数学知识,时间复杂度是一个指数复杂度。而RECURSIVE-MATRIX-CHAIN也是一个指数复杂度的,但是还有递归函数调用开销,还会重叠算某些子问题(穷举没有重叠),重复调用函数。

~\\
\noindent\textbf{15.3-2}

图太难画,略。归并排序并没有重叠子问题,每次归并,没有重复可用的操作,所以备忘技术没有用

~\\
\noindent\textbf{15.4-3}

见算法2,3

其中数组$b,c$是引用传递,可以被函数内部改变。

\begin{algorithm}
    \caption[]{MEMO-LCS-LENGTH-AUX$(X,Y,c,b)$}
    \label{alg:2}
    \begin{algorithmic}[1]
        \STATE m=$|X|$
        \STATE n=$|Y|$
        \IF{ $c[m,n]!=0$ or $m==0$ or $n==0$}
            \RETURN $c[m,n]$
        \ENDIF
        \IF{$x_m==y_n$}
            \STATE $b[m,n]=\nwarrow$
            \STATE $c[m,n]=$MEMO-LCS-LENGTH-AUX$(X[1\dots m-1],Y[1\dots n-1],c,b)$$+1$
        \ELSIF{MEMO-LCS-LENGTH-AUX$(X[1\dots m-1],Y,c,b)$$\ge$MEMO-LCS-LENGTH-AUX$(X,Y[1\dots n-1],c,b)$}
            \STATE $b[m,n]=\uparrow$
            \STATE $c[m,n]=$MEMO-LCS-LENGTH-AUX$(X[1\dots m-1],Y,c,b)$
        \ELSE 
            \STATE $b[m,n]=\leftarrow$
            \STATE $c[m,n]=$MEMO-LCS-LENGTH-AUX$(X,Y[1\dots n-1],c,b)$
        \ENDIF
        \RETURN $c[m,n]$
    \end{algorithmic}
\end{algorithm}

\begin{algorithm}
    \caption[]{MEMO-LCS-LENGTH$(X,Y)$}
    \label{alg:3}
    \begin{algorithmic}[1]
        \STATE let $b[1\dots m,1\dots n]$ and $c[1\dots m,1\dots n]$be new table
        \FOR{$i=1$ to $m$} 
            \STATE $c[i,0]=0$
        \ENDFOR
        \FOR{$i=1$ to $n$}
            \STATE $c[0,i]=0$
        \ENDFOR
        \STATE MEMO-LCS-LENGTH-AUX$(X,Y,c,b)$//$b$ and $c$,passed by reference
        \RETURN $b$ and $c$
    \end{algorithmic}
\end{algorithm}

~\\
\noindent\textbf{15.5-3}

在每次计算$e[i,j]$时都重新计算了$\omega(i,j)$,每次多出$\Theta(j-i)$次加法。

每次多出$O(n)$规模,总的时间复杂度增加到$O(n^3)$

\end{document}