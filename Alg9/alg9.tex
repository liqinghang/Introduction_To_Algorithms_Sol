\documentclass[UTF8]{ctexart}
\usepackage{algorithm}
\usepackage{algorithmic}
\usepackage{amsmath,amssymb}
\usepackage{booktabs}
\usepackage{geometry}
\usepackage{tikz}
\usepackage{color}

\geometry{a4paper,scale=0.7}

\renewcommand{\algorithmicrequire}{ \textbf{Input:}} %Use Input in the format of Algorithm
\renewcommand{\algorithmicensure}{ \textbf{Output:}} %UseOutput in the format of Algorithm

\begin{document}
    
SA22225226 李青航

\noindent\textbf{17.1-1}

不是。最极端的,当我们有连续的操作序列,
只含有MULTIPOP$(S,k)$,\\MULTIPUSH$(S,k)$
显然每次的摊还代价是$O(k)$

~\\
\noindent\textbf{17.1-2}

考虑一个极端情况,二进制是$1000...000$(有$k-1$个0),进行一个DECREMENT减一操作,翻转了
$k$~位数,再INCREMENT加一操作,又翻转了$k$~位数,连续做$n$~次这样极端操作,
时间复杂度$O(nk)$

~\\
\noindent\textbf{17.2-1}

赋予摊还代价:~PUSH~2,~POP~1,~COPY~0

每次PUSH操作,为自己缴费1元,为后面复制这个元素预先缴费1元,
因为栈最多有$k$个元素,至少都有$k$元的信用预存款够复制。
所以每次摊还代价都是常数。$n$次操作,时间复杂度$O(n)$

~\\
\noindent\textbf{17.3-1}


设$\Phi '(D_n)=\Phi (D_n)-\Phi (D_0)$

因为$\Phi (D_i)\ge \Phi (D_0)$且$\Phi (D_0) \ne 0$,

所以有$\Phi '(D_n)=\Phi (D_n)-\Phi (D_0)\ge \Phi (D_0)-\Phi (D_0)=0$

又
\begin{equation*}
    \begin{aligned}
        \sum_{i=1}^{n}\hat{c_i} 
        &=\sum_{i=1}^{n}c_i+\Phi '(D_n)-\Phi ' (D_0)\\
        &=\sum_{i=1}^{n}c_i+(\Phi (D_n)-\Phi  (D_0))-(\Phi (D_0)-\Phi  (D_0))\\
        &=\sum_{i=1}^{n}c_i+\Phi (D_n)-\Phi  (D_0)
    \end{aligned}
\end{equation*}

所以摊还代价相同

~\\
\noindent\textbf{17.3-3}

设势函数为$n\lg (n)$,$n$是最小堆的大小,最坏情况下操作的时间均为$O(\lg n)$

INSERT摊还代价是$O(\lg n)$,因为插入一次,摊还$\lg n$使得堆的大小增加1

但是抽取最小元素的摊还代价是$O(1)$,因为实际上的势函数的代价已经算过了,所以是常数级别



~\\
\noindent\textbf{17.3-4}

因为$\Phi (D_n)=s_n,~\Phi (D_0)=s _0$,并且初始有$n$个对象的栈的摊还代价是$O(n)$

由式子 17.3 得知,总代价是$O(n) + s_n - s_0$



\end{document}