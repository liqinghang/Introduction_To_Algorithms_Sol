\documentclass[UTF8]{ctexart}
\usepackage{algorithm}
\usepackage{algorithmic}
\usepackage{amsmath,amssymb}
\usepackage{booktabs}
\usepackage{geometry}
\usepackage{tikz}
\usepackage{color}

\geometry{a4paper,scale=0.7}

\renewcommand{\algorithmicrequire}{ \textbf{Input:}} %Use Input in the format of Algorithm
\renewcommand{\algorithmicensure}{ \textbf{Output:}} %UseOutput in the format of Algorithm

\begin{document}
    李青航 SA22225226

    \noindent\textbf{16.4-1 }

    根据拟阵性质定义

    1.$S$是一个有限集(已知条件)

    2.假设$k\ge 0$,就是$\mathcal{I}_k$是非空的
    为了证明遗传性,认为$A \in \mathcal{I}_k$,
    这就是说$|A|\le k$,然后如果$B\subseteq A$,
    这意味着$|B|\le |A|\le k$,所以$B\in \mathcal{I}_k$

    3.证明交换性质。让$A,B\in \mathcal{I}_k$,$|A|<|B|$
    然后选取一个任意元素 $x\in B\\A$,然后有
    $|A\cup\{x\}|=|A|+1\le |B|\le k$,
    所以我们能使$A\cup \{x\}\in \mathcal{I}_k$

    ~\\
    \noindent\textbf{16.5-1}

    
\begin{table}[h]
    \centering
    \begin{tabular}{cccccccc} 
    \toprule
    $a_i$ & 1  & 2  & 3  & 4  & 5  & 6  & 7   \\
    $d_i$ & 4  & 2  & 4  & 3  & 1  & 4  & 6   \\
    $w_i$ & 10 & 20 & 30 & 40 & 50 & 60 & 70  \\
    \bottomrule
    \end{tabular}
    \end{table}

    我们从贪婪地构造矩阵阵开始,
    首先添加成本最高的未完成任务。我们添加任务7 6 5 4 3。
    然后,为了安排任务1或2,我们需要留下未完成的更重要的任
    务。所以我们的调度是$<5 3 4 6 7 1 2>$
    总惩罚只有$w_1+w_2 = 30$

    ~\\
    \noindent\textbf{16.5-2}

    创建长度为$n$的数组$B$,初始化全为0。
    对于每个元素$a\in A$,在$B[a.deadline]$上加1。
    如果$B[a.deadline] > a.deadline$,返回集合不是独立的。
    否则,继续。如果成功检查了$A$的每个元素,返回集合是独立的。

\end{document}