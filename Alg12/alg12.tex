\documentclass[UTF8]{ctexart}
\usepackage{algorithm}
\usepackage{algorithmic}
\usepackage{amsmath,amssymb}
\usepackage{booktabs}
\usepackage{geometry}
\usepackage{tikz}
\usepackage{color}

\geometry{a4paper,scale=0.7}

\renewcommand{\algorithmicrequire}{ \textbf{Input:}} %Use Input in the format of Algorithm
\renewcommand{\algorithmicensure}{ \textbf{Output:}} %UseOutput in the format of Algorithm

\begin{document}
    SA22225226 李青航

    \noindent\textbf{26.2-2}

    26.1(b)的,横跨切割的流

    $$f(S,T)=11+1-4+7+4=19$$

    该切割的流量

    $$c(S,T)=16+4+7+4=31$$

    ~\\
    \noindent\textbf{26.2-4}

    图26.6的最大流是23,对应的最小割是$S=\{s,v_1,v_2,v_4\},T=\{v_3,t\}$

    (c)左的$v_3\rightarrow v_2$是抵消的


\end{document}