\documentclass[UTF8]{ctexart}
\usepackage{algorithm}
\usepackage{algorithmic}
\renewcommand{\algorithmicrequire}{ \textbf{Input:}} %Use Input in the format of Algorithm
\renewcommand{\algorithmicensure}{ \textbf{Output:}} %UseOutput in the format of Algorithm
% 参考:https://blog.csdn.net/jzwong/article/details/52399112
\begin{document}

李青航 SA22225226\\
\textbf{2.1-2}\\
\begin{algorithm}
    \caption{No Increasing INSERTION-SORT$(A)$}
    \label{alg:1}
    \begin{algorithmic}[1]
        \FOR{ $j=2$ \TO $A.length$}
            \STATE $key=A[j]$\
            \STATE $i=j-1$\
            \WHILE{$i>0$ and $A[i]<key$}
                \STATE$A[i+1]=A[i]$
                \STATE$i=i-1$
            \ENDWHILE
            \STATE $A[i+1]=key$
        \ENDFOR
        

    \end{algorithmic}
\end{algorithm}
\\
\textbf{2.1-3}\\
\textbf{初始化:}$j=1 ,A.length=1$时,要么$A[1]=$value,返回对应位置$i=1$\ ,要么没找到返回NIL,显然正确。\\
\textbf{保持:}不断迭代,当碰到$A[j]=v$时(第5行)返回对应位置$i$。\\
\textbf{终止:}当迭代完所有$j$后,没有找到value,返回NIL。\\
\begin{algorithm}
    \caption{  Linear-Search$(A,v)$}
    \label{alg:2}
    \begin{algorithmic}[1]
        \STATE$i=NIL$
        \FOR{$j=1$to $A.length$}
            \IF{$A[j]=v$}
               \STATE$i=j$
               \RETURN $i$
            \ENDIF
        \ENDFOR
        \RETURN $i$
    \end{algorithmic}
\end{algorithm}
\\
\textbf{2.2-1}\\
$n^3/1000-100n^2-100n+3$表示为$\Theta(n^3)$
\\
\textbf{2.2-2}\\
\begin{algorithm}
    \caption{Selection Sort$(A)$}
    \label{alg:3}
    \begin{algorithmic}[1]
        \FOR{$i=1$ to $n-1$}
            \STATE$min=i$
            \FOR{$j=i+1$ to $n$}
                \IF{$A[j]<A[min]$}
                    \STATE$min=j$
                \ENDIF
            \ENDFOR
            \STATE Swap$(A[min],A[i])$
        \ENDFOR
        
        \RETURN $i$
    \end{algorithmic}
\end{algorithm}
\\
\textbf{初始化:}当$i=1$,数组$A$中也只有一个元素时,显然已经排序,正确。\\
\textbf{保持:}当迭代到寻找第$i$小的元素时,$A[1...i-1]$已经排序好,将第$i$小的与$A[i]$交换,此时,$A[1...i]$都排序好了。\\
\textbf{终止:}当进行到$n-1$次时,前$A[1...n-1]$已经非递减有序,第$n$个就是最大,整体全部排好,终止,正确。\\
\\只用进行n-1次,因为最后一个(第 $n$个)最大就应该在最后,全部已经排序好。\\
\\最好情况$\Theta(n^2)$,最坏情况$\Theta(n^2)$\\
\\
\textbf{3.1-1}\\
存在正常量$c_1=0.5, c_2=1, n_0$,当$n\ge n_0$时,$0.5(f(n)+g(n))\le max(f(n),g(n))\le f(n)+g(n)$,符合$\Theta(f(n)+g(n))$定义\\
\\
\textbf{3.1-2}\\
根据多项式定理$\mathop{{ \left( {\mathop{{x}}\nolimits_{{1}}+\mathop{{x}}\nolimits_{{2}}+ \cdots +\mathop{{x}}\nolimits_{{m}}} \right) }}\nolimits^{{b}}=\mathop{ \sum }\limits_{{\mathop{{k}}\nolimits_{{1}}+\mathop{{k}}\nolimits_{{2}}+ \cdots +\mathop{{k}}\nolimits_{{m}}=n}}\frac{{n!}}{{\mathop{{k}}\nolimits_{{1}}! \cdot \mathop{{k}}\nolimits_{{2}}! \cdot  \cdots  \cdot \mathop{{k}}\nolimits_{{m}}!}}\mathop{ \prod }\limits_{{t=1}}^{{m}}\mathop{{\mathop{{x}}\nolimits_{{t}}}}\nolimits^{{\mathop{{k}}\nolimits_{{t}}}}
$,二项式的$b$次方展开公式,$(n+a)^b=n^b+...$其中省略的...部分是$n^b$的低阶,所以根据$\Theta$记号定义,多项式可以省略低阶,所以 $(n+a)^b=\Theta (n^b)$


\end{document}